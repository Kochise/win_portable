%%
%% This is file `xkvex1.tex',
%% generated with the docstrip utility.
%%
%% The original source files were:
%%
%% xkeyval.dtx  (with options: `xkvex1')
%% 
%% ---------------------------------------
%% Copyright (C) 2004-2014 Hendri Adriaens
%% ---------------------------------------
%%
%% This work may be distributed and/or modified under the
%% conditions of the LaTeX Project Public License, either version 1.3
%% of this license or (at your option) any later version.
%% The latest version of this license is in
%%   http://www.latex-project.org/lppl.txt
%% and version 1.3 or later is part of all distributions of LaTeX
%% version 2003/12/01 or later.
%%
%% This work has the LPPL maintenance status "maintained".
%%
%% This Current Maintainer of this work is Hendri Adriaens.
%%
%% This work consists of the file xkeyval.dtx and derived files
%% keyval.tex, xkvtxhdr.tex, xkeyval.sty, xkeyval.tex, xkvview.sty,
%% xkvltxp.sty, xkvutils.tex, pst-xkey.tex, pst-xkey.sty, xkveca.cls,
%% xkvecb.cls, xkvesa.sty, xkvesb.sty, xkvesc.sty, xkvex1.tex,
%% xkvex2.tex, xkvex3.tex and xkvex4.tex.
%%
%% The following files constitute the xkeyval bundle and must be
%% distributed as a whole: readme, xkeyval.pdf, keyval.tex,
%% pst-xkey.sty, pst-xkey.tex, xkeyval.sty, xkeyval.tex, xkvview.sty,
%% xkvltxp.sty, xkvtxhdr.tex, xkvutils.tex, pst-xkey.dtx and xkeyval.dtx.
%%
\input xkeyval

\catcode`\@=11

%% xkeyval package: TeX example
%% Hendri Adriaens

\noindent Example 1. This example demonstrates the use of the new
macros for setting keys of the `xkeyval' package. Have a look at the
source for information about this example and to understand the output.

%% Define keys.
\define@key{fami}{keyi}{key 1: #1 }
\define@key{famii}{keyii}{key 2: #1 }
\define@key{famiii}{keyiii}[test 2]{key 3: #1 }
\define@key{famiv}{keyi}{key 4: #1 }

%% Set keys from multiple families.
\setkeys{fami,famii,famiii}{keyi=test 1,keyiii}

%% Multiplely defined keys will be taken from the first family on the
%% list that defines the key. When + is used, all families
%% defining the key will set it.
\setkeys{fami,famiv}{keyi=test 3}\par
\setkeys{famiv,fami}{keyi=test 4}\par
\setkeys+{fami,famiv}{keyi=test 5}\par

%% Starred form sets keys that it can find in the families specified.
%% Keys that cannot be located will be put onto a list for later use
%% by \setrmkeys.
\setkeys*{famiv}{keyi=test 9,keyii=http://www.test10.com,
  keyiii=test 11}\par

%% Try to set the remaining keys from the previous step. The starred
%% version defines a new list in case there are still keys that cannot
%% be located in the given families. The command below does not do
%% anything since the `keyi' key has already been taken from family
%% `famiv' above.
\setrmkeys*{fami}\par

%% The normal version will complain (as \setkeys) when it cannot locate
%% a key which was previously submitted to \xsetkeys and hasn't been
%% set yet by previous commands. `xkeyval' commands can also be used to
%% set `keyval' options.
\setrmkeys{famii,famiii}\par

%% Set keys, excluding some keys.
\setkeys{fami,famii,famiii}[keyi,keyiii]{keyi=test 6,keyii=test
7,keyiii=test 8}\par

%% Test whether a key is defined in a family from the list.
\key@ifundefined{fami,famii}{keyiii}
{keyiii is not defined in family `fami' or `famii'}
{keyiii is defined in family `fami' or `famii'}

%% Define some keys and macros.
\define@cmdkey[my]{fama}[my@]{keya}[none]{}
\define@choicekey[my]{fama}{keyb}{a,b,c}[b]{correct input: #1}
\define@boolkey[my]{famb}{keyc}[true]{}
\def\testa{\setkeys[my]{fama}}
\def\testb{\setkeys[my]{famb}}
\def\testset{\setkeys[my]{fama,famb}}

%% Now \testa can only set keya, \testb can only
%% set keyb and \testset can set both.
\testa{keya}
  \my@keya\
\testa{keyb}
\testb{keyc}
  \ifmy@famb@keyc true \else false \fi
\testset{keya=test 11,keyb=c,keyc=false}
  \my@keya\space
  \ifmy@famb@keyc true \else false \fi

%% Pointer system

%% First we (re)define some keys.
\define@key{fam}{keyi}{key 1: #1 }
\define@key{fam}{keyii}{key 2: #1 }
\define@key{fam}{keyiii}[test 2]{key 3: #1 }

%% Pointers can be used to copy the value of other keys in the same
%% family. This can be done in two ways. The first example shows `on
%% the spot' value saving.
\setkeys{fam}{\savevalue{keyi}=test 12,\savevalue{keyii}=\usevalue{keyi}}
\setkeys{fam}{keyiii=\usevalue{keyii}}

%% However, you can also predefine the keys to be saved for the entire
%% document with a single command.
\savekeys{fam}{keyi,keyii}
\setkeys{fam}{keyi=test 13,keyii=\usevalue{keyi}}
\setkeys{fam}{keyiii=\usevalue{keyii}}

%% Pointers can be used in default values as well. Remember that the
%% previous example used \savekeys for keyi and keyii.
\define@key{fam}{keyi}{keyi: #1 }
\define@key{fam}{keyii}[\usevalue{keyi}]{keyii: #1 }
\define@key{fam}{keyiii}[\usevalue{keyii}]{keyiii: #1 }
\setkeys{fam}{keyi=test 14}
\setkeys{fam}{keyii}
\setkeys{fam}{keyiii}

%% Presetting keys

%% Simple example of the preset system.
\define@key{fam}{keyi}{keyi: #1 }
\define@key{fam}{keyii}{keyii: #1 }
\presetkeys{fam}{keyi=blue}{keyii=\usevalue{keyi}}
\setkeys{fam}{}\par
\setkeys{fam}{keyi=red}\par
\setkeys{fam}{keyii=red}
\presetkeys{fam}{keyi=red}{}\par
\setkeys{fam}{keyii=\usevalue{keyi}}

\bye
\endinput
%%
%% End of file `xkvex1.tex'.
