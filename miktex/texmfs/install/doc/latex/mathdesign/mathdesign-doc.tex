\documentclass[10pt]{article}

\usepackage[english]{babel}

\usepackage[euro, mdput, expert, ttscaled=true, sfscaled=true]{mathdesign}

\usepackage{textcomp}

%% -------------------------------------------------------------------

\usepackage{url}
\usepackage{texlogos}
\usepackage{xspace}

%% -------------------------------------------------------------------

% Logical styles.
\newcommand{\pkgname}[1]{%
  \textsf{#1}%
  }
\newcommand{\optname}[1]{%
  \textsf{#1}%
  }
\newcommand{\filename}[1]{%
  \texttt{#1}%
}
\newcommand{\PSfont}[1]{%
  #1%
}
\newcommand{\PSchar}[1]{%
  \texttt{#1}%
}


%% -------------------------------------------------------------------

%% Margins, etc.

%% \setlength\hoffset{-1in}
%% \setlength\voffset{-1in}

\setlength\oddsidemargin{28pt}
\setlength\textwidth{398pt}

\setlength\headsep{24pt}
\setlength\topmargin{-33pt}

\setlength\footskip{36pt}
\setlength\textheight{626pt}



%\DeclareRobustCommand{\person}[2]{#1\index{#2, #1} #2}

%% -------------------------------------------------------------------

\usepackage{hyperref}

% Scott Pakin Comprehensive code
\makeatletter
\newcommand{\notpredefinedmessage}{%
  \begin{tablenote}[*]
    Not predefined in \LaTeX.  
  \end{tablenote}
}
\newcommand{\notpredefinedmessageextra}{%
  \begin{tablenote}[*]
    Not predefined in \LaTeX.  
  \end{tablenote}
}

% Redefine the LaTeX commands that are replaced by textcomp.
% This was swiped right out of ltoutenc.dtx, but with "\text..."
% changed to "\ltext...".
\DeclareTextCommandDefault{\ltextcopyright}{\textcircled{c}}
\DeclareTextCommandDefault{\ltextregistered}{\textcircled{\scshape r}}
\DeclareTextCommandDefault{\ltexttrademark}{\textsuperscript{TM}}
\DeclareTextCommandDefault{\ltextordfeminine}{\textsuperscript{a}}
\DeclareTextCommandDefault{\ltextordmasculine}{\textsuperscript{o}}

% Define an environment in which to write a single table of symbols.  The
% environment looks a lot like a table, but it doesn't float, and it gets
% an entry in the table of contents (as a subsubsection that looks like a
% subsection), as opposed to the list of tables.
%
% The first argument is a conditional.  The table will appear only if
% the value of the conditional is true.  The second argument is the
% table's caption.

% Sometimes, we need a little more horizontal spacing, too.
\def\qqquad{\hskip3em\relax}
\def\dquad{\hskip.5em\relax}

\def\fnum@table{\textbf{\tablename}~\thetable}

\newcommand{\savecaption}[1]{\global\def\currentcaption{#1}}

\newenvironment{symtable}[1]{%
    \noindent%
    \begin{minipage}[t]{\linewidth}    % Prevent page breaks
        \begin{center}
            \addtocounter{table}{1}%
            \protected@edef\@currentlabel{\thetable}%
            \addcontentsline{toc}{subsubsection}{%
                \protect\numberline{\tablename~\thetable:}{\quad #1}}%
           \savecaption{#1}
        }{%
\@makecaption{\fnum@table}{\currentcaption}\end{center}
\end{minipage}
\bigskip
}


% Same as the above, but allows page breaks.

\newenvironment{longsymtable}[2][true]{%
  \expandafter\global\expandafter\let%
    \expandafter\ifshowsymtable\csname if#1\endcsname
  \ifshowsymtable
    \mbox{}%
    \begin{center}%
    \addtocounter{table}{1}%
    \protected@edef\@currentlabel{\thetable}%
    \addcontentsline{toc}{subsubsection}{%
      \protect\numberline{\tablename~\thetable:}{#2}}%
    \@makecaption{\fnum@table}{#2}%
    \def\lt@indexed{}%
    \let\next=\relax
  \else
    % The following was taken verbatim from verbatim.sty.
    \let\do\@makeother\dospecials\catcode`\^^M\active
    \let\verbatim@startline\relax
    \let\verbatim@addtoline\@gobble
    \let\verbatim@processline\relax
    \let\verbatim@finish\relax
    \let\next=\verbatim@
  \fi
  \next
}{%
  \ifshowsymtable
    \let\@elt=\index\lt@indexed  % Close our index ranges.
    \end{center}
    \addtocounter{table}{-1}     % Make up for longtable's counter increment.
    \vskip 8ex minus 2ex
  \fi
}


% Many tables have notes beneath them.  Define an environment in which to
% display such a note, with an optional, superscripted math symbol
% preceding it.
\newenvironment{tablenote}[1][]{
  \makebox[1em]{\ensuremath{^{#1}}}%
  \begin{minipage}[t]{0.75\textwidth}%
  \setlength{\parskip}{2ex}
}{%
  \end{minipage}%
}


\def\docttfamily{\ttfamily\small}

\def\N@opt@arg[#1]#2{$#1$ & $\Big#1$ &\docttfamily\string#2}
\def\N@no@opt@arg#1{$#1$ & $\Big#1$ &\docttfamily\string#1}
\def\N{\@ifnextchar[{\N@opt@arg}{\N@no@opt@arg}}
                                % We use \displaystyle so that variable-sized symbols will be big.

\def\Q#1{ #1{A} \dquad #1{a} &
         \docttfamily\string#1\string{A\string}\string#1\string{a\string}}

\def\QQ#1{ #1{AA} \dquad #1{aa} &
         \docttfamily\string#1\string{AA\string}\string#1\string{aa\string}}

\def\R@opt@arg[#1]#2{$#1$ & $\displaystyle#1$ &\docttfamily\string#2}
\def\R@no@opt@arg#1{$#1$ & $\displaystyle#1$ &\docttfamily\string#1}
\def\R{\@ifnextchar[{\R@opt@arg}{\R@no@opt@arg}}

\newcommand{\V}[2][]{#1 & #2 &\docttfamily\string#2}
  \def\W@opt@arg[#1]#2#3{%
      $#1{#3}$ &\docttfamily\string#2\string{#3\string}}
  \def\W@no@opt@arg#1#2{%
      $#1{#2}$ &\docttfamily\string#1\string{#2\string}}
  \def\W{\@ifnextchar[{\W@opt@arg}{\W@no@opt@arg}}
\def\DX#1{$\displaystyle#1$ &\docttfamily\string#1}
\def\X#1{$#1$ &\docttfamily\string#1}
\def\T#1{#1 &\docttfamily\string#1}

\def\Y#1{$\big#1$ & $\Bigg#1$ &\docttfamily\string#1}

\def\C#1{{\docttfamily\string#1}~($#1$)\xspace}
\def\CC#1#2{{\docttfamily\string#1}~($#2$)\xspace}
\def\CT#1{{\docttfamily\string#1}~(#1)\xspace}



% TXfonts code

\newcount\curchar \newcount\currow \newcount\curcol
\newdimen\indexwd \newdimen\tempcellwd
\setbox0\hbox{\sffamily D\kern.2em}
\indexwd=\wd0

\def\hexnumber#1{\ifcase\expandafter\ident\expandafter{\number#1} 0\or
1\or 2\or 3\or 4\or 5\or 6\or 7\or 8\or 9\or A\or B\or C\or D\or E\or
F\else ?\fi}

\def\ident#1{#1}
\def\rownumber{\sffamily\hexnumber\currow}
\def\colnumber{\sffamily\hexnumber\curcol \global\advance\curcol 1 }
\def\colnumbers{\hbox to\hsize{\global\curcol 0
  \def\1{\hbox to\cellwd{\curcol}{\hfil\colnumber\hfil}}%
  \kern\indexwd\hfil\hfil
  \1\1\1\1\hfil\hfil \1\1\1\1\hfil\hfil
  \1\1\1\1\hfil\hfil \1\1\1\1\hfil\hfil
  \kern\indexwd}%
}

\def\charnumber{\setbox0=\hbox{\char\curchar}%
  \ifdim\ht0>7.5pt\reposition
  \else\ifdim\dp0>2.5pt\reposition\fi\fi
  \box0 \global\advance\curchar1 }
\def\reposition{\setbox0=\hbox{$\vcenter{\kern1.5pt\box0\kern1.5pt}$}}
\def\measurecolwidths#1{%
  \tempcellwd\hsize \advance\tempcellwd-2\indexwd
  \advance\tempcellwd -12pt
  \divide\tempcellwd 16
  \xdef\cellwd##1{\the\tempcellwd}%
}

\def\dochartA#1{%
  \begingroup
  \global\curchar=0 \global\currow=0 \global\curcol=0
  \def\hline{\kern2pt\hrule\kern3pt }%
  \setbox0\vbox{#1%
    \def\0{\hbox to\cellwd{\curcol}{\hss\charnumber\hss}}%
    \colnumbers
    \hline
    \setrow\setrow\setrow\setrow
    \hline
    \setrow\setrow\setrow\setrow
    \hline
    \setrow\setrowX\setrow\setrowX %
%    \hline %
%%     \setrowX\setrowX\setrowX\setrowX %
    \hline %
    \colnumbers
  }%
  \null\hfil\hbox{\vbox{%
    \hbox to\hsize{\kern\indexwd
      \def\fullrule{\hfil\vrule height\ht0 depth\dp0\hfil}%
      \fullrule\kern\cellwd{0}\kern\cellwd{1}\kern\cellwd{2}\kern\cellwd{3}%
      \fullrule\kern\cellwd{4}\kern\cellwd{5}\kern\cellwd{6}\kern\cellwd{7}%
      \fullrule\kern\cellwd{8}\kern\cellwd{9}\kern\cellwd{10}\kern\cellwd{11}%
      \fullrule\kern\cellwd{12}\kern\cellwd{13}\kern\cellwd{14}\kern\cellwd{15}%
      \fullrule\kern\indexwd}%
    \kern-\ht0 \kern-\dp0 \unvbox0}}\hfil%
  \endgroup
}
\def\dochartB#1{%
  \begingroup
  \global\curchar=0 \global\currow=0 \global\curcol=0
  \def\hline{\kern2pt\hrule\kern3pt }%
  \setbox0\vbox{#1%
    \def\0{\hbox to\cellwd{\curcol}{\hss\charnumber\hss}}%
    \colnumbers
    \hline
    \setrow\setrow\setrow\setrow
    \hline
    \setrow\setrow\setrow\setrow
    \hline
    \setrow\setrow\setrow\setrow %
   \hline %
    \setrow\setrow\setrow\setrow %
    \hline %
    \colnumbers
  }%
  \vbox{%
    \hbox to\hsize{\kern\indexwd
      \def\fullrule{\hfil\vrule height\ht0 depth\dp0\hfil}%
      \fullrule\kern\cellwd{0}\kern\cellwd{1}\kern\cellwd{2}\kern\cellwd{3}%
      \fullrule\kern\cellwd{4}\kern\cellwd{5}\kern\cellwd{6}\kern\cellwd{7}%
      \fullrule\kern\cellwd{8}\kern\cellwd{9}\kern\cellwd{10}\kern\cellwd{11}%
      \fullrule\kern\cellwd{12}\kern\cellwd{13}\kern\cellwd{14}\kern\cellwd{15}%
      \fullrule\kern\indexwd}%
    \kern-\ht0 \kern-\dp0 \unvbox0}%
  \endgroup
}
\def\dochartC#1{%
  \begingroup
  \global\curchar=0 \global\currow=0 \global\curcol=0
  \def\hline{\kern2pt\hrule\kern3pt }%
  \setbox0\vbox{#1%
    \def\0{\hbox to\cellwd{\curcol}{\hss\charnumber\hss}}%
    \colnumbers
    \hline
    \setrow\setrow\setrow\setrow
    \hline
    \setrow\setrow\setrow\setrow
%%     \hline
%%     \setrow\setrow\setrow\setrow %
%%    \hline %
%%     \setrowX\setrow\setrowX\setrow %
    \hline %
    \colnumbers
  }%
  \vbox{%
    \hbox to\hsize{\kern\indexwd
      \def\fullrule{\hfil\vrule height\ht0 depth\dp0\hfil}%
      \fullrule\kern\cellwd{0}\kern\cellwd{1}\kern\cellwd{2}\kern\cellwd{3}%
      \fullrule\kern\cellwd{4}\kern\cellwd{5}\kern\cellwd{6}\kern\cellwd{7}%
      \fullrule\kern\cellwd{8}\kern\cellwd{9}\kern\cellwd{10}\kern\cellwd{11}%
      \fullrule\kern\cellwd{12}\kern\cellwd{13}\kern\cellwd{14}\kern\cellwd{15}%
      \fullrule\kern\indexwd}%
    \kern-\ht0 \kern-\dp0 \unvbox0}%
  \endgroup
}

\def\setrow{\hbox to\hsize{%
  \hbox to\indexwd{\hfil\rownumber\kern.2em}\hfil\hfil
  \0\0\0\0\hfil\hfil \0\0\0\0\hfil\hfil
  \0\0\0\0\hfil\hfil \0\0\0\0\hfil\hfil
  \hbox to\indexwd{\sffamily\kern.2em \rownumber\hfil}}%
  \global\advance\currow 1 }%

\def\setrowX{\global\advance\curchar16\global\advance\currow 1\relax}

\def\cellwd#1{20pt}% initialize

\def\measurecolwidths#1{%
  \tempcellwd\hsize \advance\tempcellwd-2\indexwd
  \advance\tempcellwd -12pt
  \divide\tempcellwd 16
  \xdef\cellwd##1{\the\tempcellwd}%
}
\def \tableA #1#2#3{\par\penalty-200 \bigskip
  \font #1=#2 \relax
  \centerline{\vbox{\hsize=29pc
    \measurecolwidths{#1}%
    \centerline{#3}%
    \medskip
    \dochartA{#1}%
}}}
\def \tableB #1#2#3{\par\penalty-200 \bigskip
  \font #1=#2 \relax
  \centerline{\vbox{\hsize=29pc
    \measurecolwidths{#1}%
    \centerline{#3}%
    \medskip
    \dochartB{#1}%
}}}
\def \tableC #1#2#3{\par\penalty-200 \bigskip
  \font #1=#2 \relax
  \centerline{\vbox{\hsize=29pc
    \measurecolwidths{#1}%
    \centerline{#3}%
    \medskip
    \dochartC{#1}%
}}}

\def\comment#1{\begingroup \small \noindent {\textit{Comment:}} #1 \endgroup}

%% \renewcommand*\descriptionlabel[1]{\hspace\labelsep
%%     \normalfont#1}

\def\FSS#1#2{\fontfamily{#1}\selectfont #1
    & \quad \fontfamily{#1}\selectfont #2}

\def\FTT#1#2{\fontfamily{#1}\selectfont #1 &
    \quad \fontfamily{#1}\selectfont #2}

\newcommand\slightsize{\@setfontsize\slightsize{27}{30}}
\makeatother

\renewcommand\sfdefault{fvs}
\renewcommand\ttdefault{fvm}

\begin{document}

\title{The \pkgname{mathdesign}  package}

\author{Paul Pichaureau\footnote{\texttt{paul.pichaureau@alcandre.net}} \\ 
{\Large \mdlogo} }

\date{\today}

\maketitle


\section{Introduction}

The package \pkgname{mathdesign} replaces all the default mathematical
fonts of \TeX\ with a complete set of mathematical fonts, designed to
be combined with a text font of your choice.  

Provided fonts cover the full family of symbol of plain \TeX{} and
\LaTeX{}, the full set of the American Mathematical Society (\AmS)
symbols, the Ralph Smith's Formal Script symbol fonts (RSFS). Some
symbol used by the package \pkgname{textcomp}. Some extra symbols are
also defined. 

More fonts will be created and shared in the future!

\subsection{Requirements}

A complete \TeX{} installation is required. In particular, the text
fonts you want to use must be already present on your system.

A \emph{recent} \TeX{} distribution is recommended (e.g. Mik\TeX{}\ 
v2.2 or later, te\TeX{} v3.0 or later) as the configuration is
really simple with the \pkgname{updmap} utility.

%% \subsection{The Math Design project}

%% The main goal of the Math Design project is to provide
%% mathematical fonts for most of the text font freely available. 

%% All comments, glyph requests, piece of advice, opinions about these
%% fonts, will be warmly welcomed.

\section{Installation}

This package alone is useless. You have to install one of the full
set of fonts available. Please consult the provided README file. It deals
with all the installation and system configuration process.

\section{Interesting features}

\begin{itemize}
        \item All the symbols are provided in normal and bold versions.
        \item Support of all \LaTeX and \AmS symbols including
    blackboard bold letters ($\mathbb Q$, $\mathbb R$, $\mathbb Z$). 
        \item Extra symbols, including  \CT\euro \C\smallin \C\intclockwise
    \C\ointclockwise \C\oiint \C\oiiint.
        \item Various greek alphabets available.
        \item Support of scaled sans serif and typewriter fonts.
\end{itemize}

\pagebreak

\section{Usage and configuration}

To use one the font in your document call the \pkgname{mathdesign}
package with the appropriate option.

\begin{center}
    \begin{tabular}{p{3.5cm}|p{5.5cm}p{3cm}}
        {\bf Text fonts} & \textbf{Option name} & \textbf{Package name }\\
       \hline \hline
       Adobe Utopia & \optname{adobe-utopia}, \optname{utopia} 
       & \pkgname{mdput} \\
       URW Garamond & \optname{urw-garamond}, \optname{garamond} 
       & \pkgname{mdugm} \\
       Bitstream Charter& \optname{bitstream-charter}, \optname{charter} 
       & \pkgname{mdbch} 
    \end{tabular}
\end{center}

In the preceding table, option on the same line are equivalent. Then,
the following lines are equivalent:
\begin{verbatim}
    \usepackage[adobe-utopia]{mathdesign}
    \usepackage[utopia]{mathdesign}
\end{verbatim}

The package tries to redefine all the commands related to the glyphs
present in the fonts. As far as I know, they work fine, but you
shouldn't use package like \pkgname{amsfonts} or \pkgname{mathrsfs} in
conjunction with \pkgname{mathdesign}. A package warning will be
emitted in such case.

Don't forget that many packages redefine the same command than
\pkgname{mathdesign} (the euro currency symbol is the worst example of
this situation). You have to take care of the possible
package clashes.

The default encoding is automatically set to T1.

\subsection{Options}

\label{sec:options}

Some \pkgname{mathdesign} options use the \pkgname{keyval} interface. As usual with  \pkgname{keyval}, any spaces between words are ignored and multiple lines are allowed. Moreover, options are order-independent.

For example, the following line asks for Bitstream Charter and upright
capitals letters :
\begin{verbatim}
    \usepackage[charter, uppercase=upright]{mathdesign}
\end{verbatim}

The following options are available:
\begin{description}
        \item[\optname{greekfamily} = <value>] three greek fonts are
    available : the default mathdesign font, \texttt{didot} which came
    from GFS Didot, and \texttt{bodoni} taken from GFS Bodoni. These two fonts
    are released by the Greek Font
    Society\footnote{\texttt{http://greekfontsociety.gr/}}.
    
    % The last font is  extracted from the commercial font Adobe Minion
    % Pro. This family is shipped with recent versions of Adobe Acrobat Reader
    
        \item[\optname{expert}] if the corresponding postscript font
    are available on your system, this option activates them. See
    section \ref{sec:sc} for more informations.

        \item[\optname{euro}] activates the \pkgname{mathdesign}
    version of the euro currency symbol (\CT{\euro}). This
    redefinition takes place \verb:\AtBeginDocument:. Default value:
    \texttt{true}.

        \item[\optname{scaled}= <value> {true}]  Scale all the
    mathdesign fonts (including math and small caps when available).
    Default value: \texttt{1.0}.

        \item[\optname{sfscaled}= \texttt{true} \emph{or}
    \texttt{false}] Use a scaled version of common sans serif fonts 
    (see explanations in section \ref{sec:scaled}). Default value:
    \texttt{true}.
 
        \item[\optname{ttscaled}= \texttt{true} \emph{or}
    \texttt{false}] Use a scaled version of common typewriter fonts
    (see explanations in section \ref{sec:scaled}). Default value:
    \texttt{true}.

%%         \item[\optname{extrasymb}] activates some extra symbols and
%%     commands. See section \ref{sec:extras}.
%%         \item[\optname{fraktur}] defines a scaled version of the \AmS{}
%%     Euler Fraktur font. This font is available \emph{via} the
%%     \verb|\mathfrak| command:
%%     \begin{quotation}
%%         \verb|$\mathfrak{AC}$| \qquad gives \qquad $\mathfrak{AC}$
%%     \end{quotation}

        \item[\optname{uppercase}\ = \texttt{upright} \emph{or}
    \texttt{italicized}] \par
    In math mode, use \texttt{upright} or \texttt{italicized}
    uppercase letters. Default value: \texttt{italicized}.

        \item[\optname{lowercase}\ = \texttt{upright} \emph{or}
    \texttt{italicized}] \par
    In math mode, use \texttt{upright} or \texttt{italicized}
    lowercase letters. Default value: \texttt{italicized}.

        \item[\optname{greekuppercase}= \texttt{upright} \emph{or}
    \texttt{italicized}] 

    In math mode, use \texttt{upright} or \texttt{italicized}
    uppercase greek letters. Default value: \texttt{upright}.
 
       \item[\optname{greeklowercase}= \texttt{upright} \emph{or}
    \texttt{italicized}] 

    In math mode, use \texttt{upright} or \texttt{italicized}
    lowercase greek letters. Default value: \texttt{italicized}.

%%        \item[\optname{mdcal}] the \verb|\mathcal| command points to
%%     the formal script letters:
%%     \begin{quotation}
%%         \verb|$\mathcal{AC}$| \qquad gives \qquad $\mathscr{AC}$
%%     \end{quotation}
    
%%     \item[\optname{cmcal}] the \verb|\mathcal| command points to
%%     a scaled version of the \TeX{} calligraphic font:
%%     \begin{quotation}
%%         \verb|$\mathcal{AC}$| \qquad gives \qquad $\mathcal{AC}$
%%     \end{quotation}
%%     this is the default behaviour. Whatever the option, the
%%     \verb|\mathscr| command produces formal script letters (i.e.
%%     \verb|$\mathscr{AC}$| gives $\mathscr{AC}$ with both options).
  
\end{description}

In french traditional typography, uppercase letters and lowercase
greek letters are not italicised contrary to the english usage.  For
example
$$ \forall t \in [0,1], \qquad (1-t)\mathrm{A} + t \mathrm{B} \in
[\mathrm{AB}] $$
$$ \mathrm{R} = a^2 + b^2, \qquad \thetaup = \arctan \frac{a}{b} \quad
\Longrightarrow \quad a\cos \alphaup + b \sin \alphaup = \mathrm{R} \cos
(\alphaup + \thetaup)$$
are the ``french'' version of
$$ \forall t \in [0,1], \qquad (1-t)A + t B \in [AB] $$
$$ R = a^2 + b^2, \qquad \theta = \arctan \frac{a}{b} \quad
    \Longrightarrow \quad a\cos \alpha + b \sin \alpha = R \cos
    (\alpha + \theta)$$

    If you want to typeset a document in the old french traditions, use
    the following options:
\begin{verbatim}
  \usepackage[uppercase=upright, greeklowercase=upright, garamond]{mathdesign}
\end{verbatim}

Please, note that upright and slanted versions of the greek letters
are always available, using commands $\verb|\alphaup|$,
$\verb|\alphait|$, etc.  (see tables \ref{tab:greekup} and
\ref{tab:greekit}).

\subsection{Small capitals and oldstyle figures}

\subsubsection{Faked small capitals}

\label{sec:sc}

It is not in the goals of the Math Design project to provide small
capitals and typographic refinements of this sort. Anyway, ``faked''
small caps are defined by default\footnote{Two new \textsf{nfss} shape
    are defined and associated with these faked small capitals :
    \texttt{\textbackslash fscshape} (variant \texttt{fsc}) for the
    upright faked small capitals and \texttt{\textbackslash ficshape}
    (variante \texttt{fix}) for the slanted faked small caps}. If you
don't load the package with the option \optname{expert} then these
small capitals will be used in your document.

\subsubsection{Commercial small capitals}

Alternatively you can buy the corresponding commercial fonts and use
them with the \pkgname{mathdesign} package.

To use commercial small capitals with the charter and utopia version of the
fonts, you must:
\begin{enumerate}
        \item Obtain the corresponding commercial fonts from your favorite font
   seller. This is the font you'll need :

   \begin{tabular}{l|ll}
       Bitstream Charter \qquad 
       & Charter Small Cap         \qquad &(bchrc8a.pfb)  \\
       & Charter Bold Small Cap    \qquad & (bchbc8a.pfb) \\[12 pt]
       Adobe Utopia \qquad 
       &  Utopia Expert Regular \qquad &    (putr8x.pfb) \\
       & Utopia Expert Bold \qquad &       (putb8x.pfb)
   \end{tabular}

You need the Windows Postscript versions of the fonts. 

    \item Rename the preceding font files. I have indicated in
parenthesis the new name of each file.

    \item Put the renamed file somewhere \TeX will be able to find
them: \verb|$TEXMF/fonts/type1/bitsrea/charter| or \verb|$TEXMF/fonts/type1/adobe/utopia| should be fine.

    \item Refresh your texmf file database, by running an utility like
\texttt{mktexlsr} or \texttt{texconfig rehash}.

    \item \textsc{That's it !} Now use the \pkgname{mathdesign}
package with the option \optname{expert}. Small caps and oldstyle
figure are available with the command \verb|\textsc{...}|.

\end{enumerate}

\textbf{Disclaimer} The preceding informations are only
\emph{indications} of a possible way to install and use commercial
products. I'm not responsible for any damage caused, in whole or in
part, by following these instructions.

Anyway, I'll try to help you the best I can to properly install any
commercial fonts you have.

\subsection{Sans serif and typewriter fonts}
\label{sec:scaled}

\begin{table}[t]
    \label{tab:scaledfont}
    \centering
    \begin{tabular}{l|l}
        \textbf{Nickname} & \textbf{Font} \\ \hline \hline
        \FSS{cmss} {Computer Modern Sans Serif} \\
        \FSS{fvs}  {Bera sans (aka Bitstream Vera Sans)} \\
        \FSS{phv}  {Adobe Helvetica} \\
%%         \FSS{uag}  {URW Gothic} \\
%%         \FSS{uhv}  {URW Helvetica} \\
        \\
        \FSS{fvm}  {Bera mono (aka Bitstream Vera Mono)} \\
        \FTT{cmtt} {Computer Modern Typewriter} \\
        \FTT{pcr}  {Adobe Courier} \\
%%         \FTT{ucr}  {URW Courier} \\
    \end{tabular}
    \caption{Scaled fonts defined.}
\end{table}

In addition to the mathematical fonts, the \pkgname{mathdesign}
package defines ``scaled'' versions of the common sans serif and
typewriter fonts.

For example, in \LaTeX, if you want to set Adobe Helvetica as your
main sans serif font, you use the following command
\begin{quotation}
    \verb|\renewcommand{\sfdefault}{phv}|
\end{quotation}
where \texttt{phv} is the name of Adobe Helvetica using Karl Berry's
fontname convention.

\renewcommand{\sfdefault}{phv}

But Adobe Helvetica will not fit well with your text font. Letters
have different heights:
\begin{quotation}
\Huge a{\slightsize\textsf{a}}%
\Huge b{\slightsize\textsf{b}}%
\Huge A{\slightsize\textsf{A}}%
\Huge e{\slightsize\textsf{e}}%
\Huge D{\slightsize\textsf{D}}
\end{quotation}

\pkgname{mathdesign} defines a scaled version of this font. This version automatically replace the normal one. So, with the option \optname{sfscaled}, the usual command
\begin{quotation}
    \verb|\renewcommand{\sfdefault}{phv}|
\end{quotation}
will give you an optically adjusted version of Adobe Helvetica:
\begin{quotation}
   \Huge a\textsf{a}b\textsf{b}A\textsf{A}c\textsf{c}D\textsf{D}
\end{quotation}
As you can see on the above example, lowercase letters have now the
same height. It is not necessary the case of uppercase letters. 

Don't expect amazing result of these feature. If you mix sans serif
and typewriter fonts in the text, then the design disparities will be
become quickly obvious.

The table \ref{tab:scaledfont} enumerates all the scaled fonts defined by
the mathdesign package. 

\subsection{Configuration file}

Each family has its own configuration file (e.g.
\filename{mdput.cfg}).  You can put in these file all the commands
that \LaTeX{} should load with the family. Consult the provided files
for more informations.

\section{More fonts and symbols}

\begin{table}[t]
    
    \centering
    \begin{tabular}{*3{ll}}
\X\alphaup      & \X\iotaup       & \X\sigmaup         \\
\X\betaup       & \X\kappaup      & \X\varsigmaup      \\
\X\gammaup      & \X\lambdaup     & \X\tauup           \\
\X\deltaup      & \X\muup         & \X\upsilonup       \\
\X\epsilonup    & \X\nuup         & \X\phiup           \\
\X\varepsilonup & \X\xiup         & \X\varphiup        \\
\X\zetaup       & \X\piup         & \X\chiup           \\
\X\etaup        & \X\varpiup      & \X\psiup           \\
\X\thetaup      & \X\rhoup        & \X\omegaup         \\
\X\varthetaup   & \X\varrhoup     \\
\\
\X \varkappaup$^{\dagger}$ & \X\digammaup$^{\dagger}$\\
\\
\\
 \X\Gammaup      & \X\Xiup      & \X\Phiup            \\
 \X\Deltaup      & \X\Piup      &  \X\Psiup           \\
 \X\Thetaup      & \X\Sigmaup   & \X\Omegaup          \\
 \X\Lambdaup     & \X\Upsilonup \\
\end{tabular}
\caption{Upright Greek Letters \label{tab:greekup}}
\end{table}
\begin{table}[t]
    
    \centering
  \begin{tabular}{*3{ll}}
\X\alphait      & \X\iotait       & \X\sigmait         \\
\X\betait       & \X\kappait      & \X\varsigmait      \\
\X\gammait      & \X\lambdait     & \X\tauit           \\
\X\deltait      & \X\muit         & \X\upsilonit       \\
\X\epsilonit    & \X\nuit         & \X\phiit           \\
\X\varepsilonit & \X\xiit         & \X\varphiit        \\
\X\zetait       & \X\piit         & \X\chiit           \\
\X\etait        & \X\varpiit      & \X\psiit           \\
\X\thetait      & \X\rhoit        & \X\omegait         \\
\X\varthetait   & \X\varrhoit \\
\\
\X \varkappait$^{\dagger}$ & \X\digammait$^{\dagger}$\\
\\
 \X\Gammait      & \X\Xiit      & \X\Phiit            \\
 \X\Deltait      & \X\Piit      &  \X\Psiit           \\
 \X\Thetait      & \X\Sigmait   & \X\Omegait          \\
 \X\Lambdait     & \X\Upsilonit \\
\end{tabular}
    \caption{Italicised Greek Letters \label{tab:greekit}}
\end{table}

\subsection{Script and fraktur alphabets}

The commands \verb|\mathfrak|, \verb|\mathscr| and \verb|\mathbb| are
defined by \pkgname{mathdesign} and have the usual meanings:
\begin{itemize}
        \item \verb|\mathfrak| for fraktur letters \emph{e.g.}
    \verb|\mathfrak{A, B, S, a, b, s}| gives $\mathfrak{A}, \mathfrak{B}, 
    \mathfrak{S}, \mathfrak{a}, \mathfrak{b}, \mathfrak{s}$ 
        \item \verb|\mathscr| for script letters \emph{e.g.}
    \verb|\mathscr{A, B, S }| gives $\mathscr{A}, \mathscr{B}, \mathscr{S}$ 
     \item \verb|\mathbb| for blackboard letters \emph{e.g.}
    \verb|\mathbb{A, B, S }| gives $\mathbb{A}, \mathbb{B}, \mathbb{S}$
\end{itemize}

\subsection{Extra symbols}\label{sec:extras}

For completeness, some commands and symbols have been added:
\begin{itemize}
        \item The command \C\iddots typesets diagonal dots, similar to
    \AmS's \C\ddots.

        \item Two new big delimiters are available,
    \CC\leftwave{\leftwave\right.} and \CC\leftevaw{\leftevaw\right.}
(and the corresponding right delimiters, of course). This is an example:
$$ \leftwave \frac{a+b+c}{3}�\rightwave $$

    \item The commands \C\utimes, \C\dtimes and \C\udtimes are
    similar to \AmS's  \C\ltimes, \C\rtimes and \C\Join.

        \item The \C\in symbol has now a small version \C\smallin,
    which can be negated (\C\notsmallin). \C\owns has also a small
    version (\C\smallowns and \C\notsmallowns).

    \item Various new integrals are defined : \C\intclockwise
    \C\ointclockwise \C\oiint \C\oiiint.

    \item The {\docttfamily\string\figurecircled} command is the
equivalent of \verb|\textcircled| circled command, but the circle is
especially designed for figures : \verb|\figurecircled{1}| gives
\figurecircled{1} (better than \verb|\textcircled{1}| : 
\textcircled{1}).

\end{itemize}

Some Text Companion symbols are also defined, including \CT\texteuro
(see table \ref{text-comp}).  To use them, you must load the
\pkgname{textcomp} package.


\subsection{Copyright notice}

The fonts provided by the Math Design project are free software;
you can redistribute it and/or modify it under the terms of the GNU
General Public License as published by the Free Software Foundation;
either version 2 of the License, or (at your option) any later
version.

This program is distributed in the hope that it will be useful, but
WITHOUT ANY WARRANTY; without even the implied warranty of
MERCHANTABILITY or FITNESS FOR A PARTICULAR PURPOSE.  See the GNU
General Public License (appendix \ref{app:gpl} of this document) for
more details.

\subsection{Acknowledgements}

I have borrowed many codes, ideas, glyphs from various sources, and I
would like to thanks all the authors of the original material, among
others Alan Jeffrey and Jeremy Gibbons (\pkgname{stmaryrd}), Yannis
Haralambous (\pkgname{yhmath} and the great greek fonts from $\Omega$),
Young Ryu (\pkgname{txfonts/pxfonts}), Antony Phan (\pkgname{mathabx})
and the \AmS.

I would like to thank in particular C\'eline Chevalier, S\'ebastien
Desvreux and Walter Appel for their kind support and friendly help.

%% horizontal brace

%% wave

% llbracket

% wide accents

%% \subsection{Technical extra}

%% Following an idea of the \pkgname{fontinst} package, the following
%% \verb|fontdimen| have been defined. These parameters concerns the text font:
%% \smallskip
%% \begin{center}
%% \begin{tabular}[t]{c|c|l}
%% \hline
%% \textbf{Parameter} & \textbf{Name} & \textbf{Description} \\\hline\hline 
%% \verb|fontdimen8| & capheight      & \quad Postscript CapHeight \\
%% \verb|fontdimen9| & ascender       & \quad Postscript Ascender \\
%% \verb|fontdimen10| & acccapheight  & \quad accented cap height \\
%% \verb|fontdimen11| & descender     & \quad Postscript Descender \\
%% \verb|fontdimen12| & maxheight     & \quad max height (from Postscript
%% FontBBox) \\
%% \verb|fontdimen13| & maxdepth      & \quad max depth (from Postscript
%% FontBBox)\\
%% \verb|fontdimen14| & digitwidth    & \quad digit width (width of the 6)\\
%% %% \verb|fontdimen15| & verticalstem  & dominant width of verical stems \\
%% \verb|fontdimen16| & baselineskip  & \quad baselineskip (12pt)
%% \end{tabular}
%% \end{center}


%% \section{Known issues and workarounds}

%% Some fonts use

\setcounter{tocdepth}{2}
\tableofcontents

\appendix

\section{Commands available}

This is a remind of all the commands redefined in the
\pkgname{mathdesign} package\footnote{The following table are strongly
    inspired from the excellent Scoot Pakin's \emph{Comprehensive
        \LaTeX{} Symbol List}
    \url{http://www.ctan.org/tex-archive/help/Catalogue/entries/comprehe
        nsive.html}}.

\begin{symtable}{Math Design extra symbols}
\label{mdbin}
\begin{tabular}{*4{cl}}
\X\smallin &
\X\smallowns &
\X\notsmallin &
\X\notsmallowns \\
\X\rightangle 
\end{tabular}

%% \bigskip
%% \notpredefinedmessage
\end{symtable}

\begin{symtable}{Variable-sized Math Design Operators}
\label{mdop}
\renewcommand{\arraystretch}{1.75}  
\begin{tabular}{*2{c@{$\:$}cl@{\qquad}}c@{$\:$}cl}
\R\intclockwise &
\R\ointclockwise &
\R\ointctrclockwise \\
\R\oiint &
\R\oiiint \\
\end{tabular}
\end{symtable}


\begin{symtable}{Binary Operators}
\label{bin}
\begin{tabular}{*4{cl}}
\X\amalg           & \X\cup          & \X\oplus    & \X\times           \\
\X\ast             & \X\dagger       & \X\oslash   & \X\triangleleft    \\
\X\bigcirc         & \X\ddagger      & \X\otimes   & \X\triangleright   \\
\X\bigtriangledown & \X\diamond      & \X\pm       & \X\unlhd$^*$       \\
\X\bigtriangleup   & \X\div          & \X\rhd$^*$  & \X\unrhd$^*$       \\
\X\bullet          & \X\lhd$^*$      & \X\setminus & \X\uplus           \\
\X\cap             & \X\mp           & \X\sqcap    & \X\vee             \\
\X\cdot            & \X\odot         & \X\sqcup    & \X\wedge           \\
\X\circ            & \X\ominus       & \X\star     & \X\wr              \\
\end{tabular}

\bigskip
\notpredefinedmessage
\end{symtable}


\begin{symtable}{Variable-sized Math Operators}
\label{op}
\renewcommand{\arraystretch}{1.75}  
\begin{tabular}{*2{c@{$\:$}cl@{\qquad}}c@{$\:$}cl}
\R\bigcap    & \R\bigotimes & \R\bigwedge       \\
\R\bigcup    & \R\bigsqcup  & \R\sum       \\
\R\bigodot   & \R\biguplus  & \R\int       \\
\R\bigoplus  & \R\bigvee    & \R\oint      \\
\R\prod      & \R\coprod  \end{tabular}
\end{symtable}


\begin{symtable}{Binary Relations}
\label{rel}
\begin{tabular}{*4{cl}}
\X\approx   & \X\equiv    & \X\perp     & \X\smile  \\
\X\asymp    & \X\frown    & \X\prec     & \X\succ   \\
\X\bowtie   & \X\Join$^*$ 
& \X\preceq   & \X\succeq \\
\X\cong     & \X\mid      & \X\propto   & \X\vdash  \\
\X\dashv    & \X\models   & \X\sim                  \\
\X\doteq    & \X\parallel & \X\simeq                \\
\end{tabular}
\bigskip
\notpredefinedmessage
\end{symtable}



\begin{symtable}{Subset and Superset Relations}
\label{subsets}
\begin{tabular}{*3{cl}}
\X\sqsubset$^*$ 
& \X\sqsupseteq & \X\supset   \\
\X\sqsubseteq   & \X\subset     & \X\supseteq \\
\X\sqsupset$^*$ 
& \X\subseteq                 \\
\end{tabular}
\bigskip
\notpredefinedmessage
\end{symtable}

\begin{symtable}{Inequalities}
\label{inequal-rel}
\begin{tabular}{*5{cl}}
\X\geq & \X\gg & \X\leq & \X\ll & \X\neq \\
\end{tabular}
\end{symtable}



\begin{symtable}{Arrows}
\label{arrow}
\begin{tabular}{*3{cl}}
\X\Downarrow          & \X\longleftarrow      & \X\nwarrow     \\
\X\downarrow          & \X\Longleftarrow      & \X\Rightarrow  \\
\X\hookleftarrow      & \X\longleftrightarrow & \X\rightarrow  \\
\X\hookrightarrow     & \X\Longleftrightarrow & \X\searrow     \\
\X\leadsto$^*$        & \X\longmapsto         & \X\swarrow     \\
\X\leftarrow          & \X\Longrightarrow     & \X\uparrow     \\
\X\Leftarrow          & \X\longrightarrow     & \X\Uparrow     \\
\X\Leftrightarrow     & \X\mapsto             & \X\updownarrow \\
\X\leftrightarrow     & \X\nearrow            & \X\Updownarrow \\
\end{tabular}
\notpredefinedmessage
\end{symtable}


\begin{symtable}{Harpoons}
\label{harpoons}
\begin{tabular}{*2{cl}}
\X\leftharpoondown   & \X\rightharpoondown   \\
\X\leftharpoonup     & \X\rightharpoonup                           \\
 \X\rightleftharpoons
\end{tabular}
\end{symtable}


\begin{symtable}{Letter-like Symbols}
\label{letter-like}
\begin{tabular}{*5{cl}}
\X\bot    & \X\forall & \X\imath & \X\ni      & \X\top \\
\X\ell    & \X\hbar   & \X\in    & \X\partial & \X\wp  \\
\X\exists & \X\Im     & \X\jmath & \X\Re               \\
\end{tabular}
\end{symtable}



\begin{symtable}{Extension Characters}
\label{ext}
\begin{tabular}{*2{cl}}
\X\relbar & \X\Relbar \\
\end{tabular}
\end{symtable}




\begin{symtable}{Variable-sized Delimiters}
\label{dels}
\renewcommand{\arraystretch}{1.75}  
\begin{tabular}{ccl@{\quad}ccl@{\quad}ccl@{\quad}ccl}
\N\downarrow & \N\Downarrow & \N{[}           & \N{]} \\
\N\langle    & \N\rangle    & \N|             & \N\| \\
\N\lceil     & \N\rceil     & \N\uparrow      & \N\Uparrow          \\
\N\lfloor    & \N\rfloor    & \N\updownarrow  & \N\Updownarrow      \\
\N(          & \N)          & \N\{           & \N\}               \\
\N/          & \N\backslash                                         \\
\end{tabular}
\end{symtable}

\begin{symtable}{Large, Variable-sized Delimiters}
\label{ldels}
\renewcommand{\arraystretch}{2.5}  
\begin{tabular}{*3{ccl@{\qquad}}ccl}
\Y\lmoustache & \Y\rmoustache & \Y\lgroup    & \Y\rgroup \\
\Y\arrowvert  & \Y\Arrowvert  & \Y\bracevert
\end{tabular}
\end{symtable}

\begin{symtable}{Math-mode Accents}
\label{math-accents}
\begin{tabular}{*4{cl}}
\W\acute{a}    & \W\check{a}    & \W\grave{a}    & \W\tilde{a} \\
\W\bar{a}      & \W\ddot{a}     & \W\hat{a}      & \W\vec{a}   \\
\W\breve{a}    & \W\dot{a}      & \W\mathring{a}               \\
\end{tabular}

\end{symtable}



\begin{symtable}{Extensible Accents}
\label{extensible-accents}
\renewcommand{\arraystretch}{1.5}
\begin{tabular}{*4l}
\W\widetilde{abc}$^*$         & \W\widehat{abc}$^*$    \\
\W\overleftarrow{abc}$^\dag$  & \W\overrightarrow{abc}$^\dag$ \\
\W\overline{abc}              & \W\underline{abc}      \\
\W\overbrace{abc}             & \W\underbrace{abc}     \\[5pt]
\W\sqrt{abc}                                   \\
\end{tabular}
\end{symtable}


\begin{symtable}{\AmS\ Extensible Accents}
\label{extensible-arrows}
\renewcommand{\arraystretch}{1.5}
\begin{tabular}{cl@{\qquad}cl}
\W\overleftrightarrow{abcde}  & \W\underleftrightarrow{abcde} \\
\W\underleftarrow{abcde}      & \W\underrightarrow{abcde}     \\
\W\xleftarrow{abcde}          & \W\xrightarrow{abcde}         \\
\end{tabular}
\end{symtable}

\begin{symtable}{Dots}
\begin{tabular}{*{3}{cl@{\hspace*{1.5cm}}}cl}
\X\cdotp & \X\colon  & \X\ldotp & \X\vdots  \\
\X\cdots & \X\ddots & \X\ldots  & \X\iddots$^*$ \\
\end{tabular}
\bigskip
\notpredefinedmessage
\end{symtable}




\begin{symtable}{Miscellaneous \LaTeX {} Symbols}
\label{ord}
\begin{tabular}{*4{cl}}
\X\aleph          & \X\Diamond$^*$    & \X\infty   & \X\prime     \\
\X\angle          & \X\diamondsuit    & \X\mho$^*$ & \X\sharp     \\
\X\backslash      & \X\emptyset       & \X\nabla   & \X\spadesuit \\
\X\Box$^{*}$      & \X\flat           & \X\natural & \X\surd      \\
\X\clubsuit       & \X\heartsuit      & \X\neg     & \X\triangle  \\
\end{tabular}
\bigskip
\notpredefinedmessage
\end{symtable}

\begin{symtable}{\LaTeX{} Commands Defined to Work in Both Math and Text Mode}
\label{math-text}
\begin{tabular}{*3{ccl@{\qqquad}}ccl}
\V\$ & \V\_              & \V\ddag    & \V\{ \\
\V\P & \V\copyright      & \V\dots    & \V\} \\
\V\S & \V\dag            & \V\pounds          \\
\end{tabular}
\end{symtable}


\begin{symtable}{Predefined \LaTeX{} Text-mode Commands}
\label{text-predef}
\begin{tabular}{ccl@{\qqquad}ccl}
\V\textasciicircum     & \V\textless            \\
\V\textasciitilde      & \V\textordfeminine    \\
\V\textasteriskcentered & \V\textordmasculine \\
\V\textbackslash       & \V\textparagraph       \\
\V\textbar             & \V\textperiodcentered  \\
\V\textbraceleft       & \V\textquestiondown    \\
\V\textbraceright      & \V\textquotedblleft    \\
\V\textbullet          & \V\textquotedblright   \\
\V\textcopyright
                       & \V\textquoteleft       \\
\V\textdagger          & \V\textquoteright      \\
\V\textdaggerdbl       & \V\textregistered      \\
\V\textdollar          & \V\textsection         \\
\V\textellipsis        & \V\textsterling        \\
\V\textemdash          & \V\texttrademark        \\
\V\textendash          & \V\textunderscore      \\
\V\textexclamdown      & \V\textvisiblespace    \\
\V\textgreater         \\
\end{tabular}
\end{symtable}

\begin{symtable}{Text-mode Accents}
\label{text-accents}
  \begin{tabular}{*3{cl@{\hspace*{3em}}}cl}
  \Q\"     & \Q\`   & \Q\k \\
  \Q\'     & \Q\b   & \Q\r \\
  \Q\.     & \Q\c   & \QQ\t \\
  \Q\=     & \Q\d   & \Q\u \\      
  \Q\^     & \Q\H   & \Q\v \\
   \Q\~    \\
  \end{tabular}
\par\medskip
\begin{tabular}{cl@{\hspace*{3em}}cl}
 \Q\textcircled
\end{tabular}
\end{symtable}

\begin{symtable}{\AmS\ Commands Defined to Work in Both Math and Text Mode}
\label{ams-math-text}
\begin{tabular}{*2{cl@{\qquad}}cl}
\X\checkmark & \X\circledR & \X\maltese
\end{tabular}
\end{symtable}

\begin{symtable}{\AmS\ Binary Operators}
\label{ams-bin}
\begin{tabular}{*3{cl}}
\X\barwedge        & \X\circledcirc     & \X\intercal        \\
\X\boxdot          & \X\circleddash     & \X\Join            \\
\X\boxminus        & \X\Cup             & \X\leftthreetimes  \\
\X\boxplus         & \X\curlyvee        & \X\ltimes          \\
\X\boxtimes        & \X\curlywedge      & \X\rightthreetimes \\
\X\Cap             & \X\divideontimes   & \X\rtimes          \\
\X\centerdot       & \X\dotplus         & \X\smallsetminus   \\
\X\circledast      & \X\doublebarwedge  & \X\veebar          \\
\end{tabular}
\end{symtable}

\begin{symtable}{\AmS\ Extra Binary Operators (see section~\ref{sec:extras})}
\label{ams-extra-bin}
\begin{tabular}{*3{cl}}
\X\utimes          & \X\dtimes          & \X\udtimes \\
\end{tabular}
\end{symtable}

\begin{symtable}{\AmS\ Variable-sized Math Operators}
\label{ams-large}
\renewcommand{\arraystretch}{1.85}  
\begin{tabular}{c@{$\:$}cl@{\qquad}c@{$\:$}cl}
\R\idotsint & \R\iiint \\[12pt]
\R\iiiint   & \R\iint  \\
\end{tabular}
\end{symtable}

\begin{symtable}{\AmS\ Binary Relations}
\label{ams-rel}
\begin{tabular}{*3{cl}}
\X\approxeq      & \X\eqcirc        & \X\succapprox    \\
\X\backepsilon   & \X\fallingdotseq & \X\succcurlyeq   \\
\X\backsim       & \X\multimap      & \X\succsim       \\
\X\backsimeq     & \X\pitchfork     & \X\therefore     \\
\X\because       & \X\precapprox    & \X\thickapprox   \\
\X\between       & \X\preccurlyeq   & \X\thicksim      \\
\X\Bumpeq        & \X\precsim       & \X\varpropto     \\
\X\bumpeq        & \X\risingdotseq  & \X\Vdash         \\
\X\circeq        & \X\shortmid      & \X\vDash         \\
\X\curlyeqprec   & \X\shortparallel & \X\Vvdash        \\
\X\curlyeqsucc   & \X\smallfrown    &                  \\
\X\doteqdot      & \X\smallsmile    &                  \\
\end{tabular}
\end{symtable}


\begin{symtable}{\AmS\ Negated Binary Relations}
\label{ams-nrel}
\begin{tabular}{*3{cl}}
\X\ncong     & \X\nshortparallel & \X\nVDash      \\
\X\nmid      & \X\nsim           & \X\precnapprox \\
\X\nparallel & \X\nsucc          & \X\precnsim    \\
\X\nprec     & \X\nsucceq        & \X\succnapprox \\
\X\npreceq   & \X\nvDash         & \X\succnsim    \\
\X\nshortmid & \X\nvdash                          \\
\end{tabular}
\end{symtable}

\begin{symtable}{\AmS\ Subset and Superset Relations}
\label{ams-subsets}
\begin{tabular}{*3{cl}}
\X\nsubseteq  & \X\subseteqq  & \X\supsetneqq    \\
\X\nsupseteq  & \X\subsetneq  & \X\varsubsetneq  \\
\X\nsupseteqq & \X\subsetneqq & \X\varsubsetneqq \\
\X\sqsubset   & \X\Supset     & \X\varsupsetneq  \\
\X\sqsupset   & \X\supseteqq  & \X\varsupsetneqq \\
\X\Subset     & \X\supsetneq                     \\
\end{tabular}
\end{symtable}

\begin{symtable}{\AmS\ Inequalities}
\label{ams-inequal-rel}
\begin{tabular}{*3{cl}}
\X\eqslantgtr  & \X\gtrless     & \X\lneq      \\
\X\eqslantless & \X\gtrsim      & \X\lneqq     \\
\X\geqq        & \X\gvertneqq   & \X\lnsim     \\
\X\geqslant    & \X\leqq        & \X\lvertneqq \\
\X\ggg         & \X\leqslant    & \X\ngeq      \\
\X\gnapprox    & \X\lessapprox  & \X\ngeqq     \\
\X\gneq        & \X\lessdot     & \X\ngeqslant \\
\X\gneqq       & \X\lesseqgtr   & \X\ngtr      \\
\X\gnsim       & \X\lesseqqgtr  & \X\nleq      \\
\X\gtrapprox   & \X\lessgtr     & \X\nleqq     \\
\X\gtrdot      & \X\lesssim     & \X\nleqslant \\
\X\gtreqless   & \X\lll         & \X\nless     \\
\X\gtreqqless  & \X\lnapprox                   \\
\end{tabular}
\end{symtable}

\begin{symtable}{\AmS\ Triangle Relations}
\label{ams-triangle-rel}
\begin{tabular}{*4{cl}}
\X\blacktriangleleft  & \X\ntriangleright   & \X\trianglerighteq  &  \\
\X\blacktriangleright & \X\ntrianglerighteq  &   \X\vartriangleleft \\ 
\X\ntriangleleft      &\X\ntrianglelefteq & \X\vartriangleright \\
\X\trianglelefteq   &  \X\triangleq     \\
\end{tabular}
\end{symtable}

\begin{symtable}{\AmS\ Arrows}
\label{ams-arrows}
\begin{tabular}{*3{cl}}
\X\circlearrowleft  & \X\leftleftarrows      & \X\rightleftarrows   \\
\X\circlearrowright & \X\leftrightarrows     & \X\rightrightarrows  \\
\X\curvearrowleft   & \X\leftrightsquigarrow & \X\rightsquigarrow   \\
\X\curvearrowright  & \X\Lleftarrow          & \X\Rsh               \\
\X\dashleftarrow    & \X\looparrowleft       & \X\twoheadleftarrow  \\
\X\dashrightarrow   & \X\looparrowright      & \X\twoheadrightarrow \\
\X\downdownarrows   & \X\Lsh                 & \X\upuparrows        \\
\X\leftarrowtail    & \X\rightarrowtail      &                      \\
\end{tabular}
\end{symtable}


\begin{symtable}{\AmS\ Negated Arrows}
\label{ams-narrows}
\begin{tabular}{*3{cl}}
\X\nLeftarrow      & \X\nLeftrightarrow & \X\nRightarrow     \\
\X\nleftarrow      & \X\nleftrightarrow & \X\nrightarrow     \\
\end{tabular}
\end{symtable}


\begin{symtable}{\AmS\ Harpoons}
\label{ams-harpoons}
\begin{tabular}{*3{cl}}
\X\downharpoonleft  & \X\leftrightharpoons                        & \X\upharpoonleft  \\
\X\downharpoonright & \X\rightleftharpoons & \X\upharpoonright \\
\end{tabular}
\end{symtable}

\begin{symtable}{\AmS\ Hebrew Letters}
\label{ams-hebrew}
\begin{tabular}{*6l}
\X\beth & \X\gimel & \X\daleth
\end{tabular}
\end{symtable}

\begin{symtable}{\AmS\ Letter-like Symbols}
\label{ams-letter-like}
\begin{tabular}{*3{cl}}
\X\Bbbk       & \X\complement & \X\hbar    \\
\X\circledR   & \X\Finv       & \X\hslash  \\
\X\circledS   & \X\Game       & \X\nexists \\
\end{tabular}
\end{symtable}

\begin{symtable}{\AmS\ Delimiters}
\label{ams-del}
\begin{tabular}{*2{cl}}
\X\ulcorner & \X\urcorner \\
\X\llcorner & \X\lrcorner
\end{tabular}
\end{symtable}

\begin{symtable}{\AmS\ Math-mode Accents}
\label{ams-math-accents}
\begin{tabular}{cl@{\hspace*{2em}}cl}
\W\dddot{a}    & \W\ddddot{a} \\
\end{tabular}
\end{symtable}


\begin{symtable}{Miscellaneous \AmS\ Symbols}
\label{ams-misc}
\begin{tabular}{*3{cl}}
\X\angle            & \X\blacktriangledown & \X\mho            \\
\X\backprime        & \X\diagdown          & \X\sphericalangle \\
\X\bigstar          & \X\diagup            & \X\square         \\
\X\blacklozenge     & \X\eth               & \X\triangledown   \\
\X\blacksquare      & \X\lozenge           & \X\varnothing     \\
\X\blacktriangle    & \X\measuredangle     & \X\vartriangle    \\

\end{tabular}
\end{symtable}

\begin{symtable}{Text Companion symbols (Not predefined in \LaTeX.  Use the package \pkgname{textcomp})}
\label{text-comp}
\begin{tabular}{*2{cl}}
\T\textbardbl &
\T\textbigcircle \\
\T\textborn &
\T\textbrokenbar \\
\T\textbullet &
\T\textcelsius \\
\T\textcent &
\T\textcentoldstyle \\ 
\T\textcopyright &
\T\textdagger \\
\T\textdaggerdbl &
\T\textdegree \\
\T\textdied &
\T\textdivorced \\
\T\textdollar &
\T\textdollaroldstyle \\
\T\textdownarrow &
\T\texteightoldstyle \\
\T\textestimated &
\T\textfiveoldstyle \\
\T\textfouroldstyle &
\T\textguarani \\ 
\T\textlbrackdbl &
\T\textleftarrow \\
\T\textlira &
\T\textmarried \\
\T\textmu &
\T\textnineoldstyle \\
\T\textnumero &
\T\textohm \\
\T\textonehalf &
\T\textoneoldstyle \\
\T\textonequarter &
\T\textopenbullet \\
\T\textordfeminine &
\T\textordmasculine \\
\T\textpertenthousand &
\T\textperthousand \\
\T\textpm &
\T\textrbrackdbl \\
\T\textregistered &
\T\textrightarrow \\
\T\textsection &
\T\textsevenoldstyle \\
\T\textsixoldstyle &
\T\textsterling \\ 
\T\textsurd &
\T\textthreeoldstyle \\
\T\textthreequarters &
\T\texttrademark \\
\T\texttwooldstyle &
\T\textuparrow \\
\T\textuparrow &
\T\textzerooldstyle \\
\T\texteuro 
\end{tabular}
\end{symtable}


\newpage

\section*{The GNU General Public License}

\label{app:gpl}

\begin{center}
{\parindent 0in

Version 2, June 1991

Copyright \copyright\ 1989, 1991 Free Software Foundation, Inc.

\bigskip

59 Temple Place - Suite 330, Boston, MA  02111-1307, USA

\bigskip

Everyone is permitted to copy and distribute verbatim copies
of this license document, but changing it is not allowed.
}
\end{center}

\begin{center}
{\bf\large Preamble}
\end{center}


The licenses for most software are designed to take away your freedom to
share and change it.  By contrast, the GNU General Public License is
intended to guarantee your freedom to share and change free software---to
make sure the software is free for all its users.  This General Public
License applies to most of the Free Software Foundation's software and to
any other program whose authors commit to using it.  (Some other Free
Software Foundation software is covered by the GNU Library General Public
License instead.)  You can apply it to your programs, too.

When we speak of free software, we are referring to freedom, not price.
Our General Public Licenses are designed to make sure that you have the
freedom to distribute copies of free software (and charge for this service
if you wish), that you receive source code or can get it if you want it,
that you can change the software or use pieces of it in new free programs;
and that you know you can do these things.

To protect your rights, we need to make restrictions that forbid anyone to
deny you these rights or to ask you to surrender the rights.  These
restrictions translate to certain responsibilities for you if you
distribute copies of the software, or if you modify it.

For example, if you distribute copies of such a program, whether gratis or
for a fee, you must give the recipients all the rights that you have.  You
must make sure that they, too, receive or can get the source code.  And
you must show them these terms so they know their rights.

We protect your rights with two steps: (1) copyright the software, and (2)
offer you this license which gives you legal permission to copy,
distribute and/or modify the software.

Also, for each author's protection and ours, we want to make certain that
everyone understands that there is no warranty for this free software.  If
the software is modified by someone else and passed on, we want its
recipients to know that what they have is not the original, so that any
problems introduced by others will not reflect on the original authors'
reputations.

Finally, any free program is threatened constantly by software patents.
We wish to avoid the danger that redistributors of a free program will
individually obtain patent licenses, in effect making the program
proprietary.  To prevent this, we have made it clear that any patent must
be licensed for everyone's free use or not licensed at all.

The precise terms and conditions for copying, distribution and
modification follow.

\begin{center}
{\Large Terms and Conditions For Copying, Distribution and
  Modification}
\end{center}


%\renewcommand{\theenumi}{\alpha{enumi}}
\begin{enumerate}

\addtocounter{enumi}{-1}

\item 

This License applies to any program or other work which contains a notice
placed by the copyright holder saying it may be distributed under the
terms of this General Public License.  The ``Program'', below, refers to
any such program or work, and a ``work based on the Program'' means either
the Program or any derivative work under copyright law: that is to say, a
work containing the Program or a portion of it, either verbatim or with
modifications and/or translated into another language.  (Hereinafter,
translation is included without limitation in the term ``modification''.)
Each licensee is addressed as ``you''.

Activities other than copying, distribution and modification are not
covered by this License; they are outside its scope.  The act of
running the Program is not restricted, and the output from the Program
is covered only if its contents constitute a work based on the
Program (independent of having been made by running the Program).
Whether that is true depends on what the Program does.

\item You may copy and distribute verbatim copies of the Program's source
  code as you receive it, in any medium, provided that you conspicuously
  and appropriately publish on each copy an appropriate copyright notice
  and disclaimer of warranty; keep intact all the notices that refer to
  this License and to the absence of any warranty; and give any other
  recipients of the Program a copy of this License along with the Program.

You may charge a fee for the physical act of transferring a copy, and you
may at your option offer warranty protection in exchange for a fee.

\item

You may modify your copy or copies of the Program or any portion
of it, thus forming a work based on the Program, and copy and
distribute such modifications or work under the terms of Section 1
above, provided that you also meet all of these conditions:

\begin{enumerate}

\item 

You must cause the modified files to carry prominent notices stating that
you changed the files and the date of any change.

\item

You must cause any work that you distribute or publish, that in
whole or in part contains or is derived from the Program or any
part thereof, to be licensed as a whole at no charge to all third
parties under the terms of this License.

\item
If the modified program normally reads commands interactively
when run, you must cause it, when started running for such
interactive use in the most ordinary way, to print or display an
announcement including an appropriate copyright notice and a
notice that there is no warranty (or else, saying that you provide
a warranty) and that users may redistribute the program under
these conditions, and telling the user how to view a copy of this
License.  (Exception: if the Program itself is interactive but
does not normally print such an announcement, your work based on
the Program is not required to print an announcement.)

\end{enumerate}


These requirements apply to the modified work as a whole.  If
identifiable sections of that work are not derived from the Program,
and can be reasonably considered independent and separate works in
themselves, then this License, and its terms, do not apply to those
sections when you distribute them as separate works.  But when you
distribute the same sections as part of a whole which is a work based
on the Program, the distribution of the whole must be on the terms of
this License, whose permissions for other licensees extend to the
entire whole, and thus to each and every part regardless of who wrote it.

Thus, it is not the intent of this section to claim rights or contest
your rights to work written entirely by you; rather, the intent is to
exercise the right to control the distribution of derivative or
collective works based on the Program.

In addition, mere aggregation of another work not based on the Program
with the Program (or with a work based on the Program) on a volume of
a storage or distribution medium does not bring the other work under
the scope of this License.

\item
You may copy and distribute the Program (or a work based on it,
under Section 2) in object code or executable form under the terms of
Sections 1 and 2 above provided that you also do one of the following:

\begin{enumerate}

\item

Accompany it with the complete corresponding machine-readable
source code, which must be distributed under the terms of Sections
1 and 2 above on a medium customarily used for software interchange; or,

\item

Accompany it with a written offer, valid for at least three
years, to give any third party, for a charge no more than your
cost of physically performing source distribution, a complete
machine-readable copy of the corresponding source code, to be
distributed under the terms of Sections 1 and 2 above on a medium
customarily used for software interchange; or,

\item

Accompany it with the information you received as to the offer
to distribute corresponding source code.  (This alternative is
allowed only for noncommercial distribution and only if you
received the program in object code or executable form with such
an offer, in accord with Subsection b above.)

\end{enumerate}


The source code for a work means the preferred form of the work for
making modifications to it.  For an executable work, complete source
code means all the source code for all modules it contains, plus any
associated interface definition files, plus the scripts used to
control compilation and installation of the executable.  However, as a
special exception, the source code distributed need not include
anything that is normally distributed (in either source or binary
form) with the major components (compiler, kernel, and so on) of the
operating system on which the executable runs, unless that component
itself accompanies the executable.

If distribution of executable or object code is made by offering
access to copy from a designated place, then offering equivalent
access to copy the source code from the same place counts as
distribution of the source code, even though third parties are not
compelled to copy the source along with the object code.

\item
You may not copy, modify, sublicense, or distribute the Program
except as expressly provided under this License.  Any attempt
otherwise to copy, modify, sublicense or distribute the Program is
void, and will automatically terminate your rights under this License.
However, parties who have received copies, or rights, from you under
this License will not have their licenses terminated so long as such
parties remain in full compliance.

\item
You are not required to accept this License, since you have not
signed it.  However, nothing else grants you permission to modify or
distribute the Program or its derivative works.  These actions are
prohibited by law if you do not accept this License.  Therefore, by
modifying or distributing the Program (or any work based on the
Program), you indicate your acceptance of this License to do so, and
all its terms and conditions for copying, distributing or modifying
the Program or works based on it.

\item
Each time you redistribute the Program (or any work based on the
Program), the recipient automatically receives a license from the
original licensor to copy, distribute or modify the Program subject to
these terms and conditions.  You may not impose any further
restrictions on the recipients' exercise of the rights granted herein.
You are not responsible for enforcing compliance by third parties to
this License.

\item
If, as a consequence of a court judgment or allegation of patent
infringement or for any other reason (not limited to patent issues),
conditions are imposed on you (whether by court order, agreement or
otherwise) that contradict the conditions of this License, they do not
excuse you from the conditions of this License.  If you cannot
distribute so as to satisfy simultaneously your obligations under this
License and any other pertinent obligations, then as a consequence you
may not distribute the Program at all.  For example, if a patent
license would not permit royalty-free redistribution of the Program by
all those who receive copies directly or indirectly through you, then
the only way you could satisfy both it and this License would be to
refrain entirely from distribution of the Program.

If any portion of this section is held invalid or unenforceable under
any particular circumstance, the balance of the section is intended to
apply and the section as a whole is intended to apply in other
circumstances.

It is not the purpose of this section to induce you to infringe any
patents or other property right claims or to contest validity of any
such claims; this section has the sole purpose of protecting the
integrity of the free software distribution system, which is
implemented by public license practices.  Many people have made
generous contributions to the wide range of software distributed
through that system in reliance on consistent application of that
system; it is up to the author/donor to decide if he or she is willing
to distribute software through any other system and a licensee cannot
impose that choice.

This section is intended to make thoroughly clear what is believed to
be a consequence of the rest of this License.

\item
If the distribution and/or use of the Program is restricted in
certain countries either by patents or by copyrighted interfaces, the
original copyright holder who places the Program under this License
may add an explicit geographical distribution limitation excluding
those countries, so that distribution is permitted only in or among
countries not thus excluded.  In such case, this License incorporates
the limitation as if written in the body of this License.

\item
The Free Software Foundation may publish revised and/or new versions
of the General Public License from time to time.  Such new versions will
be similar in spirit to the present version, but may differ in detail to
address new problems or concerns.

Each version is given a distinguishing version number.  If the Program
specifies a version number of this License which applies to it and ``any
later version'', you have the option of following the terms and conditions
either of that version or of any later version published by the Free
Software Foundation.  If the Program does not specify a version number of
this License, you may choose any version ever published by the Free Software
Foundation.

\item
If you wish to incorporate parts of the Program into other free
programs whose distribution conditions are different, write to the author
to ask for permission.  For software which is copyrighted by the Free
Software Foundation, write to the Free Software Foundation; we sometimes
make exceptions for this.  Our decision will be guided by the two goals
of preserving the free status of all derivatives of our free software and
of promoting the sharing and reuse of software generally.

\begin{center}
{\Large
No Warranty
}
\end{center}

\item
\textbf { Because the program is licensed free of charge, there is no warranty
for the program, to the extent permitted by applicable law.  Except when
otherwise stated in writing the copyright holders and/or other parties
provide the program ``as is'' without warranty of any kind, either expressed
or implied, including, but not limited to, the implied warranties of
merchantability and fitness for a particular purpose.  The entire risk as
to the quality and performance of the program is with you.  Should the
program prove defective, you assume the cost of all necessary servicing,
repair or correction.}

\item
\textbf {In no event unless required by applicable law or agreed to in writing
will any copyright holder, or any other party who may modify and/or
redistribute the program as permitted above, be liable to you for damages,
including any general, special, incidental or consequential damages arising
out of the use or inability to use the program (including but not limited
to loss of data or data being rendered inaccurate or losses sustained by
you or third parties or a failure of the program to operate with any other
programs), even if such holder or other party has been advised of the
possibility of such damages.}

\end{enumerate}


\begin{center}
{\Large End of Terms and Conditions}
\end{center}





\end{document}
